\documentclass{article}

% The preceding line is only needed to identify funding in the first footnote. If that is unneeded, please comment it out.
\usepackage{cite}
\usepackage{amsmath,amssymb,amsfonts}
\usepackage{algorithmic}
\usepackage{graphicx}
\usepackage{float}
\usepackage{textcomp}
\usepackage{xcolor}
\usepackage{outlines}
\usepackage{enumitem}
\setenumerate[1]{label=\arabic*.}
\setenumerate[2]{label=\alph*.}

\def\BibTeX{{\rm B\kern-.05em{\sc i\kern-.025em b}\kern-.08em
    T\kern-.1667em\lower.7ex\hbox{E}\kern-.125emX}}
\begin{document}
\begin{titlepage}
    \begin{center}
        \vspace{4cm}
        \large
        \textbf{
            Individual Meeting \\
            Dylan Miracle \\
            ICS 698-02 \\
            Spring 2021 \\
            April 6, 2021 \\
            Dr. Jigang Liu
        }
    \end{center}
\end{titlepage}
\title{Individual Meeting}

\author{Dylan Miracle\\
\textit{Department of Computer Science} \\
\textit{Metropolitan State University}\\
St. Paul, Minnesota, USA \\
dylan.miracle@my.metrostate.edu
}

\maketitle

\tableofcontents

\section{What is the current status?}
\subsection{What has been done?}
I have developed a simple example and workflow for getting started using a quantum computer. The paper has a demonstration of how to build a quantum circuit and how a quantum circuit can modeled with an intermediate representation (QASM). It is shown how to use the quantum framework in a well known programming language (python). Finally I have mapped out the workflow that takes you through the execution of a quantum code.

\subsection{What has been partially done?}
A comparison of quantum and classical programming. This is difficult because everything you do with quantum code looks like classical code, but the fundamentals are completely different. 

Additionally I have started on the references, but need to spend more time formatting the references correctly. 

\subsection{What has not been done yet and why?}
Benchmarking has not been done and will probably be abandoned. The initial scope of the project was too broad and the timeline was too optimistic. It will require learning a good deal more about quantum computers to be able to implement benchmarks.

\section{What is your plan to complete the term paper?}
\subsection{What are the steps for completing the term paper?}
Complete writing the examples. Complete comparison of classical and quantum computing. 

\subsection{What is the timeline?}
Week of April 10: Draft a comparison of digital computation and quantum computing. \\
Week of April 17: Complete references. \\
Week of April 24: Revise the paper and slides. \\

\subsection{What is the plan B?}
No plan B because the paper only needs to be completed and no additional research will be done.

\section{What are the constraints or limitations you are currently facing in finishing up the term paper?}
\subsection{What are the resources you are still looking for?}
None

\subsection{What are the problems you are not sure whether solutions can be found in 2 to 3 weeks?}
Create a coherent explanation of the similarities and differences between quantum computation. 

\subsection{What are the assumptions for the conclusions you made in your term paper?}
Current software engineers will need a way to anchor their understandings of quantum computing using classical computing. 

\end{document}
