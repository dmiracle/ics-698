\documentclass{article}

% The preceding line is only needed to identify funding in the first footnote. If that is unneeded, please comment it out.
\usepackage{cite}
\usepackage{amsmath,amssymb,amsfonts}
\usepackage{algorithmic}
\usepackage{graphicx}
\usepackage{textcomp}
\usepackage{xcolor}
\def\BibTeX{{\rm B\kern-.05em{\sc i\kern-.025em b}\kern-.08em
    T\kern-.1667em\lower.7ex\hbox{E}\kern-.125emX}}
\begin{document}
\begin{titlepage}
    \begin{center}
        \vspace{4cm}
        \Huge
        \textbf{
            Quantum theory, the Church-Turing principle and the universal quantum computer \\
            D. Deutch \\
            Proceedings of the Royal Society of London \\
            1985 \\
            Dylan Miracle \\
            ICS 698-02 \\
            Spring 2021 \\
            Feb 17, 2021 \\
            Dr. Jigang Liu
        }
    \end{center}
\end{titlepage}
\title{Quantum theory, the Church-Turing principle and the universal quantum computer}

\author{Dylan Miracle\\
\textit{Department of Computer Science} \\
\textit{Metropolitan State University}\\
St. Paul, Minnesota, USA \\
dylan.miracle@my.metrostate.edu
}

\maketitle
\section{Background}
Computing machines have an input and output state, and in the classical world the output is a direct and deterministic function of the input. Two machines are classically computationally equivalent if the same input produces the same output. Quantum machines however do not have such functions and the output state is not directly observable, only projections of the output state onto a computational basis can be measured and must have a probabilistic interpretation.

\section{Main idea/conclusion}

This paper extends the Church-Turing hypothesis to include physical laws of quantum mechanics. Where as the Church-Turing hypothesis says that a Turing machine can be extended to any possible calculation, the quantum computer is hypothesized to act under natural laws, discrete time, and finite means.

\section{Support facts/algorithms/methods}


\section{Arguments/disagreements/concerns}


\section{Interesting findings}


\section{Quotations}
Every finitely realizible physical system can be perfectly simulated by a universal model computing machine operating by finite means.

. . . it is consistent with our present knowledge of the interactions present in Nature that every real (dissipative) finite physical system can be perfectly simulated by the universal quantum computer Q.
\end{document}
