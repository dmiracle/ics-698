\documentclass{article}

% The preceding line is only needed to identify funding in the first footnote. If that is unneeded, please comment it out.
\usepackage{cite}
\usepackage{amsmath,amssymb,amsfonts}
\usepackage{algorithmic}
\usepackage{graphicx}
\usepackage{textcomp}
\usepackage{xcolor}
\def\BibTeX{{\rm B\kern-.05em{\sc i\kern-.025em b}\kern-.08em
    T\kern-.1667em\lower.7ex\hbox{E}\kern-.125emX}}
\begin{document}
\begin{titlepage}
    \begin{center}
        \vspace{4cm}
        \large
        \textbf{
            A Survey of Quantum Programming Languages: History, Methods, and Tools \\
            Donald A. Sofge \\
            Proceedings of the Second International Conference on Quantum, Nano, and Micro Technologies \\
            2008 \\
            Dylan Miracle \\
            ICS 698-02 \\
            Spring 2021 \\
            Feb 17, 2021 \\
            Dr. Jigang Liu
        }
    \end{center}
\end{titlepage}
\title{A Survey of Quantum Programming Languages: History, Methods, and Tools}

\author{Dylan Miracle\\
\textit{Department of Computer Science} \\
\textit{Metropolitan State University}\\
St. Paul, Minnesota, USA \\
dylan.miracle@my.metrostate.edu
}

\maketitle
\section{Background}
This paper is old in the world of quantum computing, but it gives a look at the early days of quantum language development. Theoretical work has been done on quantum information since Von Neuman in the 1930's, but actual computational frameworks did not begin developing unit the work of Feynman in 1982 and Deutsch in 1985. Deutsch proposed a qyantum Turing machine, a model that is used to show an example of a quantum algorithm that has substantial speedup over its classical counterpart.
\section{Main idea/conclusion}
Quantum programming languages, like classical languages, can be imperative or functional. 

The gate model of quantum algorithms is the most developed. 

\section{Support facts/algorithms/methods}
\section{Arguments/disagreements/concerns}
\section{Interesting findings}
\section{Quotations}

Quantum programming languages may be taxonomically divided into (A) imperative quantum programming languages, (B) functional quantum programming languages, and (C)others (may include mathematical formalisms not intended for computer execution).

The difficulties in formulating useful, effective, and in some sense universally capable quantum programming languages arise from several root causes. First, quantum mechanics itself (and by extension quantum information theory) is incomplete. Specifically missing is a theory of measurement. Quantum theory is quite successful in describing the evolution of quantum states, and even in predicting probabilistic outcomes after measurements have been made, but the process of state collapse is (with a few exceptional cases) not covered. So issues such as decoherence, diffusion, entanglement between particles (or entangled state, of whatever physical instantiation), and communication (including teleportation) are not well defined from a quantum information (and by extension quantum computation) perspective. Work with semantic formalisms and linear logic attempt to redress this by providing a firmer basis in a more complete logic consistent with quantum mechanics.

\end{document}
