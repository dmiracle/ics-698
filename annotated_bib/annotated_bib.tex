\documentclass[conference]{IEEEtran}
\IEEEoverridecommandlockouts
% The preceding line is only needed to identify funding in the first footnote. If that is unneeded, please comment it out.
\usepackage{cite}
\usepackage{amsmath,amssymb,amsfonts}
\usepackage{algorithmic}
\usepackage{graphicx}
\usepackage{textcomp}
\usepackage{xcolor}
\def\BibTeX{{\rm B\kern-.05em{\sc i\kern-.025em b}\kern-.08em
    T\kern-.1667em\lower.7ex\hbox{E}\kern-.125emX}}
\begin{document}
\begin{titlepage}
    \begin{center}
        \vspace{4cm}
        \Huge
        \textbf{
            Annotated Bibliography \\
            Dylan Miracle \\
            ICS 698-02 \\
            Spring 2021 \\
            Feb 17, 2021 \\
            Dr. Jigang Liu
        }
    \end{center}
\end{titlepage}
\title{Survey of Quantum Programming Frameworks: Annotated Bibliography}

\author{\IEEEauthorblockN{Dylan Miracle}
\IEEEauthorblockA{\textit{Deptartment of Computer Science} \\
\textit{Metropolitan State University}\\
St. Paul, Minnesota, USA \\
dylan.miracle@my.metrostate.edu
}}

\maketitle

\begin{abstract}
This document is a model and instructions for \LaTeX.
This and the IEEEtran.cls file define the components of your paper [title, text, heads, etc.]. *CRITICAL: Do Not Use Symbols, Special Characters, Footnotes, 
or Math in Paper Title or Abstract.
\end{abstract}

\begin{IEEEkeywords}
    component, formatting, style, styling, insert
\end{IEEEkeywords}

\section{Introduction}
Quantum computing has fundamental implications for computation. Researchers are continuing to expand the understanding of what quantum computers can do while 

\section{Term Paper Subtopics}

\subsection{Overview of Quantum Computing}


Quantum Circuit Synthesis using Group Decomposition and Hilbert Spaces

    Author(s): Michael S. Saraivanov \\
    Publisher: MS, Portland State University \\
    Date: 2013 \\
    Pages: 165 \\
    Relevance Quantum computing is currently at the gate level -- when classical computing was at the level of ands, ors, nots etc. This thesis is a comprehensive overview of all the quantum operations that could be available to a quantum computer.


\subsection{Quantum Circuits}

\subsection{QASM}

\subsection{Python Frameworks}


\subsection{Implementation}


\section{Conclusion}


\end{document}
