\documentclass{article}

% The preceding line is only needed to identify funding in the first footnote. If that is unneeded, please comment it out.
\usepackage{cite}
\usepackage{amsmath,amssymb,amsfonts}
\usepackage{algorithmic}
\usepackage{graphicx}
\usepackage{textcomp}
\usepackage{xcolor}
\def\BibTeX{{\rm B\kern-.05em{\sc i\kern-.025em b}\kern-.08em
    T\kern-.1667em\lower.7ex\hbox{E}\kern-.125emX}}
\begin{document}
\begin{titlepage}
    \begin{center}
        \vspace{4cm}
        \Huge
        \textbf{
            Progress Report III \\
            Dylan Miracle \\
            ICS 698-02 \\
            Spring 2021 \\
            Mar 2, 2021 \\
            Dr. Jigang Liu
        }
    \end{center}
\end{titlepage}
\title{Getting started coding quantum computers with QASM and Qiskit}

\author{Dylan Miracle\\
\textit{Department of Computer Science} \\
\textit{Metropolitan State University}\\
St. Paul, Minnesota, USA \\
dylan.miracle@my.metrostate.edu
}

\maketitle

\tableofcontents

\section{Preliminary findings}
Quantum computing has a rich theoretical background but practical quantum computers have only recently become available to researchers and business users. These computers are still rare and operated by a small number of hardware vendors. The dominant paradigm for programing quantum computers is the gate model of quantum computing that defines a sequence of operations on a set of qubits. In this paper we introduce a fundamental quantum circuit and demonstrate its implementation using an intermediate representation, QASM, and a higher level quantum computing framework Qiskit.

\section{Tables/figures/charts}

\section{Preliminary conclusion}

\section{Two arguments or reasons differentiating results}
Much work has been done to develop quantum computing frameworks. These frameworks are racing forward to increase functionality and capabilities for advanced users. This paper should give an overview of some of the fundamentals of quantum computing for an advanced undergraduate or software developer without a background in quantum mechanics.

\section{Two facts showing work is non-trivial}

\end{document}
